\documentclass[french]{article}
\usepackage[T1]{fontenc}
\usepackage[utf8]{inputenc}
\usepackage{lmodern}
\usepackage[a4paper]{geometry}
\usepackage{babel}
\usepackage{graphicx}
\usepackage{url}

\title{Devoir Maison de Modélisation Phonétique}
\begin{document}
\maketitle


Ce DM est à rendre au plus tard pour le \textbf{dimanche 20 décembre}. Tout jour de retard entraînera la perte d'un point. 
\section{Exercice 1}

Répondez aux questions ci-dessous:

\begin{enumerate}
	\item Qu’est-ce que l’intonation ? Quels sont ses quatre niveaux possibles ?
	\item  De manière générale, que signifie une intonation montante ? Une intonation descendante ?
\item Quelle est la différence entre une onde périodique et une onde apériodique ? Quelle est la différence entre une onde simple et une onde complexe ? 

	\item Qu'est-ce qu'une harmonique ? Au vue de la nature des harmoniques, est-il possible de trouver des sons sans harmonique ?
	\item A quels traits distinctifs correspondent respectivement les formants F1, F2 et F3 ? 
	\item Qu'est-ce que le Voice Onset Time ? Donnez un cas où celui-ci est positif et un autre où il est négatif.
	\item Sur un spectrogramme, qu'elle est la particularité des constrictives ?
	
	
\end{enumerate}

\section{Exercice 2}

Ouvrez le fichier "Ex\_2.png". Celui-ci contient les images de quatre ondes sinusoïdales. A partir de ces images, déterminez la période puis la fréquence de chaque onde. Calculez ensuite les quatre premières harmoniques de chacune de ces ondes. Indiquez toutes ces valeurs dans le tableau ci-dessous:

\begin{table}[h!]
	\begin{tabular}{|l|l|l|l|l|l|l|}
		\hline
		Onde & Période & Fréquence & H1 & H2 & H3 & H4 \\ \hline
		1 &         &           &    &    &    &    \\ \hline
		2 &         &           &    &    &    &    \\ \hline
		3 &         &           &    &    &    &    \\ \hline
		4 &         &           &    &    &    &    \\ \hline
	\end{tabular}
\end{table}


\newpage
\section{Exercice 3}

Avec Audacity, ouvrez le fichier "Ex\_3.mp3". Avant de commencer l'exercice, passer le mode de visualisation sur "Spectrogramme" (voir Annexe pour passer en vue Spectrogramme). Vous devriez avoir ceci à l'écran : 

\begin{figure}[h!]
	\includegraphics[width=\linewidth]{resultat.png}

\end{figure}


Ce fichier contient quatre mots. A l'aide de l'outil d'annotation d'Audacity, faites la transcription phonétique de ces mots. En plus de la transcription, vous indiquerez les allongements, les liaisons et les assimilations. Faites bien attention à faire une annotation différente pour chaque phonème, et non pas une annotation pour le mot entier. 

Une fois l'annotation terminée, \textbf{enregistrer le projet Audacity dans le même dossier que celui de ce DM}. Vous devriez avoir un fichier "Ex\_3.aup" ainsi qu'un dossier "Ex\_3\_data".


Pour rappel, pour annoter une zone dans Audacity: sélectionnez la zone avec la souris que vous voulez annoter puis faites CTRL-B (ou CMD-B sur Mac). Cela ouvrira une zone de texte dans la piste inférieure. 

Pour transcrire sur l'ordinateur, vous pouvez utiliser l'outil TypeIt, disponible ici : https://ipa.typeit.org/full/

\newpage
\section{Exercice 4}

\begin{itemize}
	\item Reportez les valeurs du tableau ci-dessous sur le fichier \textbf{grille\_ex\_4.png}
	\item reliez les points dans cet ordre \textbf{(f, d), (d, k), (k, c), (c, f)}.
\end{itemize}


 A quoi correspond cette forme, et les valeurs que vous avez tracé ? Aidez vous de la charte de l'\textbf{Alphabet Phonétique International} pour répondre à cette question. Une fois que vous avez trouvé la réponse, aidez vous de la charte pour remplacer les valeurs a, b, c, ... par les véritables valeurs correspondantes.

\begin{table}[h!]
	\begin{tabular}{|l|l|l|}
		\hline
		Valeur & X    & Y   \\ \hline
		a      & 1961 & 365 \\ \hline
		b      & 1750 & 300 \\ \hline
		c      & 1256 & 684 \\ \hline
		d      & 764  & 315 \\ \hline
		e      & 1391 & 517 \\ \hline
		f      & 2064 & 308 \\ \hline
		g      & 1417 & 381 \\ \hline
		h      & 1718 & 530 \\ \hline
		i      & 793  & 383 \\ \hline
		j      & 998  & 531 \\ \hline
		k      & 1000  & 684 \\ \hline

	\end{tabular}
\end{table}






\end{document}
